\problemname{Amalgamated Artichokes}

\illustration{0.42}{artichokes}{Picture by Hans Hillewaert via Wikimedia Commons}%
Fatima Cynara is an analyst at Amalgamated Artichokes (AA).  As with any
company, AA has had some very good times as well as some bad ones.
Fatima does trending analysis of the stock prices for AA, and she wants to
determine the largest decline in stock prices over various time spans.  For example, if over a span of time the stock
prices were $19$, $12$, $13$, $11$, $20$ and $14$, then the
largest decline would be $8$ between the first and fourth price.  If the last price
had been $10$ instead of $14$, then the largest decline would have
been $10$ between the last two prices.

Fatima has done some previous analyses and has found that the stock
price over any period of time can be modelled reasonably accurately
with the following equation:
\[
	\operatorname{price}(k) = p \cdot (\sin(a \cdot k+b) + \cos(c \cdot k+d) + 2)
\]
where $p$, $a$, $b$, $c$ and $d$ are constants.
Fatima would like you to
write a program to determine the largest price decline over a given sequence of prices.  Figure~\ref{fig:art fig} illustrates the price function for Sample Input~1.  You have to consider the prices only for integer values of $k$.

\begin{figure}[!h]
  \centering
  \includegraphics[width=0.9\textwidth]{sample1}
  \caption{Sample Input 1.  The largest decline occurs from the fourth to the seventh price.}
  \label{fig:art fig}
\end{figure}

\section*{Input}

The input consists of a single line containing $6$ integers $p$ ($1 \le p \le
1\,000$), $a$, $b$, $c$, $d$  ($0 \le a, b, c, d \le 1\,000$) and $n$ ($1 \le n \le 10^6$).
The first $5$ integers are described above.
The sequence of stock prices to consider is 
$\operatorname{price(1)}, \operatorname{price(2)}, \ldots,
\operatorname{price}(n)$.
%% The price $n$
%% indicates the number of stock prices to consider, starting with
%% $\operatorname{price(1)}, \operatorname{price(2)}, \ldots,
%% \operatorname{price}(n)$.

\section*{Output}

Display the maximum decline in the stock prices.
If there is no decline, display the number $0$.
Your output should have an absolute or relative
error of at most $10^{-6}$.

%% , and the two points
%% $x_1$ and $x_2$ where
%% $\operatorname{price}(x_2)-\operatorname{price}(x_1) =$ maximum decline.
%% If there is more than one correct answer, output the one with the
%% minimum $x_1$ price.

